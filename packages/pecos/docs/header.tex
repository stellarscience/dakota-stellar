\documentclass[letterpaper]{book}

%\usepackage{a4wide}
\oddsidemargin  0.25 in
\evensidemargin 0 in
\topmargin  0 in
\textwidth  6.25 in
\textheight 8.25 in

\usepackage{makeidx}
\usepackage{fancyhdr}
\usepackage{graphicx}
\usepackage{float}
\usepackage{alltt}
\usepackage{times}
\ifx\pdfoutput\undefined
\usepackage[ps2pdf,
            pagebackref=true,
            colorlinks=true,
            linkcolor=blue
           ]{hyperref}
\else
\usepackage[pdftex,
            pagebackref=true,
            colorlinks=true,
            linkcolor=blue
           ]{hyperref}
\fi

\usepackage{doxygen}
% Override some settings from doxygen.sty:
\rfoot[\fancyplain{}{\bfseries\scriptsize PECOS Version 1.0 Users Manual generated on \today}]{}
\lfoot[]{\fancyplain{}{\bfseries\scriptsize PECOS Version 1.0 Users Manual generated on \today}}

\makeindex
\setcounter{tocdepth}{1}
\renewcommand\footrulewidth{0.4pt}

\begin{document}

\begin{titlepage}
\pagenumbering{arabic}
\setcounter{page}{3}
\begin{center}
{\large SAND2008-xxxx}\\
{\large Unlimited Release}\\
%{\large xxxx 2008}\\
{\large \today}\\

\vspace*{1.5cm}
{\LARGE PECOS, A Parallel Environment for Creation of Random Field and
Stochastic Process Realizations}\\
\vspace*{1cm}
{\LARGE Version 1.0 Users Manual}\\
\vspace*{1cm}

{\large \bf Michael S. Eldred, Eric T. Phipps, Clayton G. Webster}\\
{\large Optimization and Uncertainty Estimation Department}\\
\vspace*{0.5cm}
{\large \bf Richard V. Field, Jr.}\\
{\large ... Department}\\
\vspace*{0.5cm}
{\large Sandia National Laboratories}\\
{\large P.O. Box 5800}\\
{\large Albuquerque, New Mexico 87185}\\
\vspace*{1cm}

{\Large \bf Abstract}
\end{center}

The PECOS (Parallel Environment for Creation Of Stochastics) library
generates samples of random fields (RFs) and stochastic processes
(SPs) from a set of user-defined power spectral densities (PSDs).  The
RF/SP may be either Gaussian or non-Gaussian and either stationary or
nonstationary, and the resulting sample is intended for run-time query
by parallel finite element simulation codes.

This report serves as a Users Manual for the PECOS software and
describes the PECOS class hierarchies and their interrelationships.
It derives directly from annotation of the actual source code and
provides detailed class documentation, including all member functions
and attributes.

\end{titlepage}

\cleardoublepage
\tableofcontents
\cleardoublepage
% Comment out lines:
%   \chapter{PECOS Page Documentation}
% Edit:
%   none at this time
% ------------------------- End header-dev.tex -------------------------
